\documentclass{article}

\usepackage[utf8]{inputenc}
\usepackage[english]{babel}
\usepackage[]{amsthm}
\usepackage[]{amssymb}
\usepackage{mathtools}
\usepackage{multicol}

\usepackage{graphicx}
\graphicspath{ {./media/} }

\setlength{\parindent}{0pt}

\DeclarePairedDelimiter\ceil{\lceil}{\rceil}
\DeclarePairedDelimiter\floor{\lfloor}{\rfloor}

\title{
    \textbf{\Large Universidad Nacional Mayor de San Marcos}

    \textbf{\large Facultad de Ingeniería de Sistemas e Informática}

    \vspace{0.7cm}

    \includegraphics[scale=0.3]{./logo.png}

    \vspace{0.7cm}

    \text{\normalsize Matemática discreta: Ecuaciones lógicas}

    \vfill
    \textbf{\footnotesize Alumno:}
    \text{\footnotesize Rodrigo José Alva Sáenz}

    \textbf{\footnotesize Docente:}
    \text{\footnotesize Dr. Ciro Rodríguez Rodríguez}

    \date{\small \today}
}

\begin{document}
\maketitle %This command prints the title based on information entered above

\newcommand{\Ejercicio}[2]{
    \section*{Ejercicio #1}

    {#2}

    \bigskip
    \textbf{Resolución:}
    \bigskip
}

\newcommand{\limplies}{\rightarrow}

\newpage

\Ejercicio{1}{
    Se sabe que únicamente P es verdadero, ¿Qué puede afirmarse del valor de verdad de cada una de las proposiciones siguientes?:
    \begin{multicols}{3}
        \begin{itemize}
            \item $P \land Q$
            \item $R \lor P$
            \item $R \land P$
            \item $S \lor \lnot P$
            \item $R \limplies P$
            \item $P \limplies Q$
            \item $P \limplies P \lor S$
            \item $\lnot P \limplies Q \land R$
            \item $S \limplies \lnot P$
            \item $R \limplies (S \limplies P)$
            \item $P \lor S \limplies (\lnot Q \limplies P)$
            \item $Q \land \lnot P \limplies R \land Q$
        \end{itemize}
    \end{multicols}
}

Reemplazando $P$ por su valor de verdad definido ($V$), obtenemos:

\begin{enumerate}
    \item $V \land Q \equiv Q$
    \item $R \lor V \equiv V$
    \item $R \land V \equiv R$
    \item $S \lor \lnot V \equiv S \lor F \equiv S$
    \item $R \limplies V \equiv V$
    \item $V \limplies Q \equiv Q$
    \item $V \limplies V \lor S \equiv V \limplies (V \lor S) \equiv V \limplies V \equiv V$
    \item $\lnot V \limplies Q \land R \equiv F \limplies (Q \land R) \equiv V$
    \item $S \limplies \lnot V \equiv S \limplies F \equiv \lnot S$
    \item $R \limplies (S \limplies V) \equiv R \limplies V \equiv V$
    \item $V \lor S \limplies (\lnot Q \limplies V) \equiv V \limplies V \equiv V$
    \item $Q \land \lnot V \limplies R \land Q \equiv (Q \land F) \limplies (R \land Q) \equiv F \limplies (R \land Q) \equiv V$
\end{enumerate}

\newpage

\Ejercicio{2}{
    Desarrollar las tablas de verdad asociadas a las siguientes

    \begin{multicols}{2}
        \begin{enumerate}
            \item $(p \land q) \limplies (r \lor \lnot p) \land r$
            \item $(\lnot p \limplies (q \land p)) \leftrightarrow \lnot q$
            \item $(\lnot p \land (q \lor r)) \limplies ((p \lor r) \land q)$
            \item $[\lnot (p \lor q) \leftrightarrow (\lnot q \land \lnot p)] \lor p$
            \item $(q \limplies \lnot r) \land s$
            \item $(p \land r) \lor (p \land q)$
        \end{enumerate}
    \end{multicols}
}

1. $(p \land q) \limplies (r \lor \lnot p) \land r$

\begin{center}
    \begin{tabular}{| c | c | c | c |}
        \hline
        $p$ & $r$ & $q$ & $(p \land q) \limplies (r \lor \lnot p) \land r$ \\
        \hline
        0 & 0 & 0 & True\\
        0 & 0 & 1 & True\\
        0 & 1 & 0 & True\\
        0 & 1 & 1 & True\\
        1 & 0 & 0 & True\\
        1 & 0 & 1 & False\\
        1 & 1 & 0 & True\\
        1 & 1 & 1 & True\\
        \hline
    \end{tabular}

    \bigskip

    La expresión es una contingencia
\end{center}

\vspace{0.5cm}

2. $(\lnot p \limplies (q \land p)) \leftrightarrow \lnot q$

\begin{center}
    \begin{tabular}{| c | c | c |}
        \hline
        $p$ & $q$ & $(\lnot p \limplies (q \land p)) \leftrightarrow \lnot q$ \\
        \hline
        0 & 0 & False \\
        0 & 1 & True \\
        1 & 0 & True \\
        1 & 1 & False \\
        \hline
    \end{tabular}

    \bigskip

    La expresión es una contingencia
\end{center}


\newpage

3. $(\lnot p \land (q \lor r)) \limplies ((p \lor r) \land q)$

\begin{center}
    \begin{tabular}{| c | c | c | c |}
        \hline
        p & r & q & $(\lnot p \land (q \lor r)) \limplies ((p \lor r) \land q)$ \\
        \hline
        0 & 0 & 0 & True \\
        0 & 0 & 1 & False \\
        0 & 1 & 0 & False \\
        0 & 1 & 1 & True \\
        1 & 0 & 0 & True \\
        1 & 0 & 1 & True \\
        1 & 1 & 0 & True \\
        1 & 1 & 1 & True \\
        \hline
    \end{tabular}

    \bigskip

    La expresión es una contingencia
\end{center}

\vspace{0.5cm}

4. $[\lnot (p \lor q) \leftrightarrow (\lnot q \land \lnot p)] \lor p$

\begin{center}
    \begin{tabular}{| c | c | c | c |}
        \hline
        p & q & $[\lnot (p \lor q) \leftrightarrow (\lnot q \land \lnot p)] \lor p$ \\
        \hline
        0 & 0 & True \\
        0 & 1 & True \\
        1 & 0 & True \\
        1 & 1 & True \\
        \hline
    \end{tabular}

    \bigskip

    La expresión es una tautología
\end{center}

\vspace{0.5cm}

5. $(q \limplies \lnot r) \land s$

\begin{center}
    \begin{tabular}{| c | c | c | c |}
        \hline
        r & q & s & $(q \limplies \lnot r) \land s$ \\
        \hline
        0 & 0 & 0 & False \\
        0 & 0 & 1 & True \\
        0 & 1 & 0 & False \\
        0 & 1 & 1 & True \\
        1 & 0 & 0 & False \\
        1 & 0 & 1 & True \\
        1 & 1 & 0 & False \\
        1 & 1 & 1 & False \\
        \hline
    \end{tabular}

    \bigskip

    La expresión es una contingencia
\end{center}

\newpage

6. $(p \land r) \lor (p \land q)$

\begin{center}
    \begin{tabular}{| c | c | c | c |}
        \hline
        p & r & q & $(p \land r) \lor (p \land q)$ \\
        \hline
        0 & 0 & 0 & False \\
        0 & 0 & 1 & False \\
        0 & 1 & 0 & False \\
        0 & 1 & 1 & False \\
        1 & 0 & 0 & False \\
        1 & 0 & 1 & True \\
        1 & 1 & 0 & True \\
        1 & 1 & 1 & True \\
        \hline
    \end{tabular}

    \bigskip

    La expresión es una contingencia
\end{center}






\end{document}

