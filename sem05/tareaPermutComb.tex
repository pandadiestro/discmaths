\documentclass{article}

\usepackage[utf8]{inputenc}
\usepackage[english]{babel}
\usepackage[]{amsthm}
\usepackage[]{amssymb}
\usepackage{mathtools}

\setlength{\parindent}{0pt}

\DeclarePairedDelimiter\ceil{\lceil}{\rceil}
\DeclarePairedDelimiter\floor{\lfloor}{\rfloor}

\title{
    \author{Rodrigo José Alva Sáenz}
    \date{\small \today}
}

\begin{document}
\maketitle %This command prints the title based on information entered above

\newpage

\section*{Ejercicio 1}

Queremos codificar los símbolos alfanuméricos (28 letras y 10 cifras) en palabras de
una cierta longitud $k$ de un alfabeto binario $\mathbb{A} = \{0; 1\}$. ¿Cuál es la mínima longitud
necesaria para poderlo hacer?

\bigskip

\textbf{Resolución:}

Considerando que tenemos $n=2$ dígitos posibles para representar 38 símbolos distintos (por lo menos) entre números y letras, tenemos que:

\begin{center}
    $PR_{n=2}^{k} \geq 38$

    \medskip

    $2^k \geq 38$

    \medskip

    $k_{min} = 6 / k \in \mathbb{N}$
\end{center}

Esto se puede comprobar porque si queremos representar 38 valores únicos en un conjunto de cifras entre 0 y 1 entonces estamos usando un número binario de la forma:

$$000000...$$

Si queremos representar por lo menos el número 38 podemos pasar este a binario:

$$\log_2(38) \approx 5.2$$

Por lo que necesitamos $\ceil{5.2} = 6$ bits para representar 38: $100110$

\bigskip

$\therefore$ La mínima longitud $k$ es 6.

\section*{Ejercicio 2}

¿Es suficiente con palabras de hasta 4 de longitud para representar todas las letras del
alfabeto ordinario en lenguaje Morse (el lenguaje Morse dispone sólo de dos símbolos:
punto y raya)?

\bigskip

\textbf{Resolución:}

Podemos obtener el número máximo de palabras que se pueden formar con 4 palabras de 2 posibles símbolos cada caracter de la siguiente forma:

$$PR^{4}_{2} = 2^4 = 16$$

\bigskip

Sin embargo, debemos considerar que podemos también construir palabras de 1, 2 y 3 caracteres de longitud. Al final, podemos considerar:

$$\sum\limits^{4}_{k=1} PR^{k}_{2} = \sum\limits^{4}_{k=1} 2^k = 2 \cdot \dfrac{1 - 2^4}{1 - 2} = 2 (15) = 30$$

\bigskip

$\therefore$ Considerando que podemos armar hasta 30 ($\geq 27$) palabras diferentes con hasta 4 caracteres en Morse, la notación si es suficiente.

\newpage

\section*{Ejercicio 3}

¿De cuántas maneras se pueden escoger tres números del 1 al 9 de manera que no salgan
dos consecutivos?

\bigskip

\textbf{Resolución:}



\end{document}

